\documentclass[a4paper,10pt]{report}
\usepackage[left=2cm,top=2cm,right=2cm,nohead,nofoot]{geometry}
\usepackage[T2A]{fontenc}
\usepackage{ucs}
\usepackage[utf8x]{inputenc}
\usepackage[english,bulgarian]{babel}
\usepackage{indentfirst}
%\usepackage{hyperref}
\usepackage{xcolor}
\usepackage{listings}
\usepackage{graphicx}
\usepackage[unicode, colorlinks=true, naturalnames=true, linkcolor=blue]{hyperref}
\usepackage{ifpdf}
\usepackage{eso-pic}

\ifpdf
  \DeclareGraphicsRule{*}{mps}{*}{}
\fi

\title{Тест по нещо}
\date{\today}

\begin{document}
%Notes:
%	трябва не трябва -> не трябва 
%	някви омазани безкрайности има някъде


\lstset{
	inputencoding=utf8x, 
	extendedchars=\true,
	basicstyle=\small\ttfamily,
	keywordstyle=\color{blue!50!black},
	commentstyle=\color{gray},
	tabsize=4,
	showstringspaces=false,
	language=Java,
	backgroundcolor=\color{red!20!blue!8!white},
%	frameround=tttt,
	frame=shadowbox,
%	framexleftmargin=5mm,
	framerule=1pt,
	rulesepcolor=\color{red!20!blue!40!white}
}

\ClearShipoutPicture
\AddToShipoutPicture{\includegraphics[width=3cm,height=4.5cm]{color-pooh.png}}

\AddToShipoutPicture{\includegraphics[width=\paperwidth,height=\paperheight]{pooh2.png}}


%\newcommand{\code}[1]{%
%	\lstset{basicstyle=\ttfamily}%
%	\lstinline{#1}%
%	\lstset{basicstyle=\tiny\ttfamily}%
%}
\newcommand{\code}[1]{\texttt{#1}}
\newcommand{\noop}{ }
\newcommand{\todo}[1]{}
%use to enable english hyphenation
\newcommand{\en}[1]{\selectlanguage{english}#1\selectlanguage{bulgarian}} 





\newcommand{\bs}{\char '134}

\newcommand{\ig}{ѝ }	


%\maketitle

%\setcounter{tocdepth}{3}
%\tableofcontents

%\newenvironment{\test}{
%\begin{enumerate}
%}
%{
%\end{enumerate}
%}


% mest commands - the generator recognizes them only on begginning of a line!

\newenvironment{test}
	{\begin{enumerate}}
	{\end{enumerate}}


\newcommand{\startQuestion}{
	\item
}

%you may want endQuestion, but it wont work :)
\newcommand{\finishQuestion}{
	\end{enumerate}
}

\newcommand{\answers}{
\begin{enumerate} 
\setlength{\topsep}{-2mm} 
\setlength{\itemsep}{0mm} 
}



\newcommand{\question}[2]{
\startQuestion
	#1
\answers
	#2
\finishQuestion

}

\newcommand{\wrong}{\item}
\newcommand{\correct}{\item(+) }
\newcommand{\ts}{\ \hspace{5.5mm}\ }

\newcommand{\postCorrect}{\item}
\newcommand{\postWrong}{\item}
\newcommand{\tableExtra}{}

%%%%%%%%%%%%%%%%%%%%%%%%%%%%%%%%%%%%%%%%%%%%%%%%%%%%%%%%%%%%%%%%%%%%%%%%%%%%%%%
% mest instructions are here
% !!max_questions=25!! -- how many questions will the variants have
% !!max_answers=5!! -- how many answers will each question have
% !!file_prefix=mestology-sample1-!! -- all generated file names begin with this prefix
%		dont know what will happen if you put directory here
% !!variants=8!! -- how many variants will be present
%		
%%%%%%%%%%%%%%%%%%%%%%%%%%%%%%%%%%%%%%%%%%%%%%%%%%%%%%%%%%%%%%%%%%%%%%%%%%%%%%%


\begin{center}
{\Large\textbf{Местология}} %the name of the course

Тест 0 – 32 юни 2007 % the name/date of the test

Вариант !!variant!! % which variant will be present
\end{center} 
% the information to be filled from the student
Трите имена: ....................................................................................................   Курс: ..... Фак. номер: ....................
% the table to fill answers. Note that in generated answers \tableExtra is replaced with the answers
\begin{tabular}{|@{}c@{}|@{}c@{}|@{}c@{}|@{}c@{}|@{}c@{}|@{}c@{}|@{}c@{}|@{}c@{}|@{}c@{}|@{}c@{}|@{}c@{}|@{}c@{}|@{}c@{}|@{}c@{}|@{}c@{}|@{}c@{}|@{}c@{}|@{}c@{}|@{}c@{}|@{}c@{}|@{}c@{}|@{}c@{}|@{}c@{}|@{}c@{}|@{}c@{}|@{}c@{}|}
\hline
&1&2&3&4&5&6&7&8&9&10&11&12&13&14&15&16&17&18&19&20&21&22&23&24&25\\
\hline
 \tableExtra
\hline
!!v!!&\ts&\ts&\ts&\ts&\ts&\ts&\ts&\ts&\ts&\ts&\ts&\ts&\ts&\ts&\ts&\ts&\ts&\ts&\ts&\ts&\ts&\ts&\ts&\ts&\ts\\
\hline
\end{tabular} 


\setlength{\parindent}{0mm}
\setlength{\parskip}{0cm} 
\setlength{\baselineskip}{0cm} 
\setlength{\itemsep}{-1cm} 


\begin{test}
\question{Местологията е:}{
	\wrong част от комплексния анализ.
	\wrong супер наука.
	\wrong ултра-мега-токат звеска наука.
	\wrong наука за глобалното затопляне.
	\wrong наука за геополитическите конфликти резултиращи от масовата балканизация на канарските острови.
	\wrong част от СЕП.
	\wrong NP-пълна.
	\wrong NP-празна.
	\correct измислена наука по името на генератора.
	\correct менте наука.
	\correct наука за генератора mest.
}
\question{Mest генератора изисква:}{
	\wrong .Net framework.
	\wrong Microsoft Visual Studio 2009
	\wrong Windows Vista.
	\wrong NakovTestGenerator.
	\wrong Microsoft Office.
	\wrong да има vi и да няма emacs.
	\wrong да има emacs и да няма vi.
	\wrong DOS 6.22.
	\wrong Python.
	\wrong нито едно от посочените.
	\correct Perl.
	\correct pdflatex.
	\correct Perl и pdflatex.
}
\question{Въпросите в един mest test започват с: }{
	\wrong списък.
	\wrong таблица.
	\wrong с \texttt{{\bs}mestQuest}
	\wrong с \texttt{{\bs}Question}
	\wrong с \texttt{{\bs}quest}
	\wrong с \texttt{Question.}
	\wrong празен ред
	\wrong \texttt{{\bs}q}
	\correct с \texttt{{\bs}question}
	\correct нито едно от посочените.
}
\question{
Какъв е резултатът от{$ \displaystyle \sum_{i=1}^{3} i $ }?
}{
	\wrong  1
	\wrong  2
	\wrong  3
	\wrong  4
	\wrong  5
	\correct 6
	\correct $\frac{12}{2}$
	\correct $2+2^2$
}
\startQuestion
	Даден е следният програмен фрагмент:

\begin{lstlisting}
public static void main(String[] args){
	i ate that fish;
}

\end{lstlisting}

Това лошо ли е?

\answers
	\wrong не
	\wrong това е страхотно
	\wrong това рути
	\wrong съвсем не
	\wrong изобщо
	\correct да
	\correct много
	\correct зависи
\finishQuestion
\question{
	Дали ще стане теста над една страница?
}{
	\wrong не
	\wrong нито едно от изброените
	\correct да
}
\question{
	Кой направи mest?
}{
	\wrong Иван
	\wrong Георги
	\wrong Станчо
	\wrong Stancho
	\wrong Мишо
	\wrong Пешо
	\correct Мило
	\correct този който написа това
}
\question{
	менте?
}{
	\wrong не
	\wrong false
	\correct да
	\correct true
}
\question{
	менте?
}{
	\wrong не
	\wrong false
	\correct да
	\correct true
}
\question{
	менте?
}{
	\wrong не
	\wrong false
	\correct да
	\correct true
}
\question{
	менте?
}{
	\wrong не
	\wrong false
	\correct да
	\correct true
}
\question{
	менте?
}{
	\wrong не
	\wrong false
	\correct да
	\correct true
}
\question{
	менте?
}{
	\wrong не
	\wrong false
	\correct да
	\correct true
}
\question{
	менте?
}{
	\wrong не
	\wrong false
	\correct да
	\correct true
}
\question{
	менте?
}{
	\wrong не
	\wrong false
	\correct да
	\correct true
}
\question{
	менте?
}{
	\wrong не
	\wrong false
	\correct да
	\correct true
}
\question{
	менте?
}{
	\wrong не
	\wrong false
	\correct да
	\correct true
}
\question{
	менте?
}{
	\wrong не
	\wrong false
	\correct да
	\correct true
}
\question{
	менте?
}{
	\wrong не
	\wrong false
	\correct да
	\correct true
}
\question{
	менте?
}{
	\wrong не
	\wrong false
	\correct да
	\correct true
}
\question{
	менте?
}{
	\wrong не
	\wrong false
	\correct да
	\correct true
}
\question{
	менте?
}{
	\wrong не
	\wrong false
	\correct да
	\correct true
}
\question{
	менте?
}{
	\wrong не
	\wrong false
	\correct да
	\correct true
}
\question{
	менте?
}{
	\wrong не
	\wrong false
	\correct да
	\correct true
}
\question{
	менте?
}{
	\wrong не
	\wrong false
	\correct да
	\correct true
}
\question{
	менте?
}{
	\wrong не
	\wrong false
	\correct да
	\correct true
}
\question{
	менте?
}{
	\wrong не
	\wrong false
	\correct да
	\correct true
}
\question{
	менте?
}{
	\wrong не
	\wrong false
	\correct да
	\correct true
}
\end{test}


Време за работа: 75 минути. \footnote{!!note!!}


\end{document}


